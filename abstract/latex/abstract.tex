\documentclass[12pt]{article}
\usepackage[a4paper, total={6in, 10in}]{geometry}


\author{Miguel P Xochicale\\
map479@bham.ac.uk \\
Department of Electronic Engineering\\
School of Engineering\\
University of Birmingham, UK}
\title{
eMOTION: Analysis of Emotion Variability and Movement Variability in the Context of Human-Robot Interaction
} 
\date{\today}

\begin{document}
\maketitle

Movement variability is an inherent feature within and between persons.
Research on measurement and understanding of movement variability has been well
established in the previous three decades in areas such as biomechanics,
sport science, psychology, cognitive science, neuroscience and robotics.
With that in mind, we hypothesise that the subtle variations of face emotions
and simple body movements can be both described and quantified in a similar 
fashion as with the methodologies of movement variability.
Such methodologies are based on nonlinear dynamics, particularly with the use of
the state space reconstruction theorem where dynamics of an unknown system
can be reconstructed using one dimensional time series.
For this work, we explain how the state space reconstruction theorem works and
present preliminary results of the use of the state reconstruction
to understand the relationship between the variability of arm movements, head
pose estimation and the emotion variability of six participants in the context of 
human-robot interactions.

The results of the state space reconstruction in the context of face emotions 
lead us to conclude that not only the variability of upper body movement 
can be analysed and quantified using the state space reconstruction theorem
but also the subtle variability of face emotion transitions across time 
(e.g. from excitement to neutral to boredom, etc)
can be understood and measured using nonlinear dynamics.
With that in mind, we can conclude that having a better understanding of 
nonlinear dynamics tools in the context of human-robot interaction 
can enhance the development of better diagnostic tools for various 
pathologies which can be applied in areas of rehabilitation, entertainment or 
sport science.
\end{document}
